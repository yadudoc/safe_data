\section{Introduction}


% Describe data science and a lab

While science has long been collaborative and data driven, today, more than ever before, researchers are dependent on the ability to safely store, effectively manage, and quickly analyze vast amounts of
data. Moreover, individual data-sets that are of interest to science and scholarship vary widely in size, often take on heterogeneous structures, and command varying levels of sensitivity. In fact, data sensitivity is often `custom' -defined specifically by
data use agreements that can be so granular as to vary from user to user and institution to institution. At the very same time, however, a typical research setting involves collaborators from multiple institutions that often share data, analysis techniques, and results. This is, in fact, a cornerstone of modern research. Yet, when data are particularly sensitive and their use agreements limit access and activities at the level of individual users, research collaboration can either lead to inadvertent breaches of agreement or can grind to a halt under administrative burden.


% Describe data and security concerns

Data sensitivity and security concerns are well founded. While many data-sets are public the most
valuable ones are often considered private due to legal or contractual obligations under which the data
is shared, or even governmental regulations due to the nature of the data. The limited and proprietary nature
of the data further increases the perceived value to the researcher. However this also adds the burden of
ensuring the security of the data to the researchers.

% Ability to compute effectively and securely.

Here are some critical problems researchers face in this space:

\begin{itemize}
\item Data-sets are large, posing problems for storage, compute and cost.
\item Varying degrees of sensitivity in data-sets require configurable security
\item Share access, analyses across individuals and groups.
\end{itemize}





