\section{Introduction}


% Describe data science and a lab

While science has long been collaborative and data driven, today, more than ever before,
researchers are dependent on technologies that enable their ability to safely store,
effectively manage, and quickly analyze vast amounts of
data. Moreover, individual datasets that are of interest to science and scholarship
vary widely in size, often take on heterogeneous structures, and command varying
levels of sensitivity. In fact, data sensitivity is often `custom'---defined specifically by
data use agreements that can be so granular as to vary from user to user and institution
to institution. At the same time, however, a typical research setting involves
collaborators from multiple institutions that often share data, analysis techniques,
and results. This is, in fact, a cornerstone of modern research. Yet, when data are
particularly sensitive and their usage agreements limit access and research activities at the
level of individual users, research collaboration can either lead to inadvertent
breaches of agreement or can grind to a halt under administrative friction.

% Describe data and security concerns

At the same time, data sensitivity and security concerns are well founded. While many datasets are public, the most
valuable ones are often considered `private' or `protected' due to legal or contractual obligations under which the data
is shared, or even, the most extreme cases, under governmental regulation. However, the limited and proprietary nature
of particular datasets can further increase perceived value to researchers given that the space of possible
questions and possible answers that can be derived from the data are, in a sense, hidden from competition.
Simultaneously, increasing sensitivity adds to the burden of ensuring data security.

% Ability to compute effectively and securely.

To address these concerns, myriad computational infrastructures
have been developed. These infrastructures, including \NAME \cite{babuji2016cloud},
implement the infrastructure and services required to securely store and analyze protected
data. However, in almost every case, there is one significant weakness to these systems: the individual humans tasked with administering and managing
data collections are fallible and prone to error. Most often, the policies and processes for managing
the data are ad hoc and lack software support.

To address these challenges we introduce two abstractions:
the `safe collection' to define the scope of a data collection
and the associated policies, and the role of a data `steward'
to manage safe collections. These two constructs allow us
to define, manage, and audit access, use, and export of
sensitive data. In this paper, we describe the implementation of these constructs in \NAME and
outline how such implementation enhances security without compromising collaborative agility.


%Here are some critical problems researchers face in this space:
%
%\begin{itemize}
%\item Datasets are large, posing problems for storage, compute and cost.
%\item Varying degrees of sensitivity in data-sets require configurable security
%\item Share access, analyses across individuals and groups.
%\end{itemize}
