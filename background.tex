\section{Background and Motivation} \label{sec:background}

There are two major camps that try to attempt to solve the problem of computing on sensitive
datasets without disclosing the underlying sensitive micro-data. Data enclaves attempt to
provide a secure environment to apply computation over sensitive data with security enforced
via controlling what enters and exits the system often via human oversight. Formal privacy
techniques on the other hand state privacy assumptions and guarantees and the privacy loss
from analyses are formally defined.

Formal privacy techniques such as homomorphic encryption, secure multiparty computation and
other cryptographic solutions for protecting private data often comes at heavy computational
overheads \cite{naehrig2011can}. Work on database based solutions \cite{popa2011cryptdb}
demonstrates the need for matching the privacy protecting methods to the analysis in question
to reduce overheads. The majority of the work in this area is focussed on the security of
numerical data, while the data we are dealing with is mostly large volumes of textual data.
Recent work \cite{kepner2014computing} on applying privacy techniques to securely compute
mathematical operation at big data scales incur a 2x overhead.

Computing systems in some of the most critical and sensitive areas such as military, avionics,
nuclear are protected from intrusion by physical isolation\cite{byres2013air}, \cite{ross2013security}.
The U.S. Census Bureau to this day has several enclaves that host sensitive microdata accessible
only on site \cite{rdc_uscensus}. With physical proximity being a limiting factor, data enclaves
evolved to support network access via encrypted channels and limited IO bandwidths \cite{lane2008using}, \cite{grossman2016toward}.
Accessibility in this case comes at the cost of increased attack surface. Such systems, while
capable of limiting the extent of data that can be viewed cannot protect against say a single
sensisitive line of microdata being viewed over a secure remote session. They depend on non-disclosure
agreements with users to prevent such information from being divulged.

Traditional data enclaves hosted privately on-premise often have massive costs arising
from the cost of physical infrastructure to secure hardware, dedicated security and networking
teams to build and operate custom software and are limited in their ability to scale
compute capabilities in response to requirements. For a distributed research team that
focusses on large scale data analysis with highly sporadic nature an on-premise data enclave
is far too expensive and is lacks accessibility..A cloud hosted system would offer the scalability,
democratize access in terms of enabling anyone to host a secure data enclave and collaborate safely with a
distributed set of peers.

Cloud Kotta is a data-enclave designed to enable computation over sensitive data-sets
in a cost effective and secure manner. In prior work we've shown that Cloud Kotta offers
a scalable solution to match the compute needs of a distributed research network.
This model involved two classes of users: 1) the administrators who take on the role of managing
users, approving privileges and handling the transfer of data onto the data stores. 2) the analysts
who perform the role of developing analytical pipelines and running analyses on the data-sets.

On Cloud Kotta we currently archive over 10TB of compressed text data and in the past year
alone, analyses on the datasets consumed over 0.25M core hours. With a cloud hosted system
such as ours, designed to serve large scale computational analyses over large datasets,
the overheads of formal privacy techniques are financially impractical. More importantly
these techniques often don't lend themselves to arbitrary text data and complex analyses
on it. The system thus protects the hosted microdata by leveraging the strong security
guarantees of the cloud vendors infrastructure and by utilizing strong access control
models. The datasets come from various sources such as JSTOR, Web of Science, IEEE,
and UChicago grants database with varying sensitivities and legal requirements to
protect the microdata.

In data-science datasets with varying levels of sensitivity and can broadly be
categorized into three \cite{ist_dataclass} :
\begin{itemize}
\item \textbf{Public} : Datasets that are in the public domain and there is no
  reasonable expectation of sensitivity within it.
\item \textbf{Confidential}: Dataset contains private information and there is a legal/contractual obligation to control access and usage.
\item \textbf {Regulated}: Dataset contains sensitive data that can be widely damaging if disclosed.
  These are tightly controlled by government regulations. Eg HIPAA, SSNs etc
\end{itemize}

The workflow of an analyst on Cloud Kotta involves requesting access to a data-set, submitting analyses
tasks via the various interfaces to interrogate the data and tune the analyses until results are generated.
These results are available by requesting the UUID4 url of the submitted task and not tied to any user.
This was designed to allow for sharing of analyses codes and results, but this introduces the risk
of unintended, unapproved access to the generated results. More importantly since there is no disclosure
controls on the results generated, any sensitive data that makes it to the results can be exported.
This work extends our current security model and extends support for the most sensitive datasets
that require closer oversight on export of results as well as access.
