\section{Related Work}

Secure data enclaves have a long and varied history starting with air gapped systems that are physically
disconnected from the internet through to increasingly common cloud-based systems.
While there are many examples of data enclaves, to the best of our knowledge none provide 
the stewardship and policy-based models used in \NAMENS. We briefly review
a range of data enclaves. 

NORC~\cite{lane2008using}, a research institution at the University of Chicago, operates a
data enclave to support researchers' investigations into programmatic, business, and policy decisions. 
NORC stores and manages a broad set of sensitive data in their FISMA compliant data enclave. The enclave
was designed with the vision of supporting multi-researcher collaboration via remote access terminals.
All microdata is hosted in a secure NORC server. 
This common approach to developing enclaves ensures security by access control, and limiting computation
to the server. However, it is limited by storage and computational infrastructure and the need to
host and operate infrastructure. It is typically not well
suited for large scale data analysis. Furthermore, users could potentially read microdata through 
their terminal, thus this approach only guarantees the privacy of the bulk of microdata.

The ICPSR data enclave \cite{icpsr} is a data enclave that hosts sensitive datasets
for social science research. While the majority of the datasets are public, ICPSR
also manages restricted use datasets such as crime
data that are protected in data enclaves. They offer two types of data enclave: physical
and virtual. ]kyle{Yadu - is this true?} The physical data enclave is a single protected server that is disconnected
from the internet and is accessible only in person. The virtual enclave is a remote desktop solution that is designed
to prevent copying of data. Systems such as this offer security at the cost of accessibility and ease of use and at
the same time have no solution to the fundamental problem of users manually copying sensitive microdata.

The National Center for Health Statistics (NCHS) Research Data Center (RDC) \cite{cdc} hosts a large collection of
restricted datasets. The datasets contain health information and are subject to HIPAA guidelines.
The data is accessible on-premise at the NCHS RDC or the Federal Statistical RDC, or in some
cases via remote access. 
%several datasets are not available for remote access. 
RDC restricts access to direct identifiers such as name and 
social security number while leaving indirect identifiers such as geography accessible. Even with the constraints
on access and limited accessibility, access to potentially identifying information leaves this system open to
linkage attacks. This is mitigated to some extent by strong vetting of research proposals.

The NCI Genomic Data Commons \cite{grossman2016toward} hosts several Petabytes of
genomic data, and provides an on-premise cloud model to enable computation on these sensitive datasets.
While the on-premise infrastructure meets compliance requirements, this choice leads to added costs in terms of
building and maintaining production compute infrastructure---an approach that is unlikely to be 
broadly available to a wide range of scientists. \NAMENS, and other cloud-based enclaves, benefit
significantly from the low-cost cloud resources made possible due to providers' economies of scale.




